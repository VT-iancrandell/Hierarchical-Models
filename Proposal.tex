\documentclass[11pt]{article}
\usepackage[pdftex]{color, graphicx}
\usepackage{amsmath, amsfonts, amssymb, mathrsfs}
\usepackage{dcolumn}
\usepackage{natbib}
\bibpunct{(}{)}{;}{a}{}{,}

\oddsidemargin=0.25in
\evensidemargin=0.25in
\textwidth=6in
\textheight=8.75in
\topmargin=-.5in
\footskip=0.5in

\date{October 25, 2014}
\author{Ian Crandell, Andy Hoegh}
\title{Hierarchical Modeling Project Proposal}
\begin{document}

\maketitle
\noindent
This work originates from a LISA project from a few years ago. At that point a simple, but not entirely satisfactory analysis was conducted. Now equipped with more sophisticated modeling techniques, we are able to conduct a more thorough analysis. The dataset was collected by the Virginia Department of Game and Inland Fisheries and presented to the Conservation Management Institute at Virginia Tech. CMI in turn solicited help from LISA with the statistical analysis.\\
\\
The VDGIF is interested in factors effecting fox survival during fox hunting trials conducted by hunt clubs. In these trials the foxes are collared, the collar cease working if a fox is killed in a trial. The dataset contains trials spanning 17 months, recording death/survival of the foxes as well the number of dogs allowed in the pen on a given trial day. The policy question revolves around setting limits on the density of dogs (e.g. dogs/acre) and time a fox is allowed to acclimate to a pen before allowing dogs. The goal isn't avoid fatalities, but rather establish a set of practices that are \emph{more} humane.\\
\\
Statistically the analysis will be quite interesting. At it's core it will be a discrete survival analysis, looking at the effect of dogs/acre and fox acclimation time, but will also require establishing a baseline survival rate (on non-trial days). Our hope is to prep the manuscript for eventual submission to a journal like \emph{The Annals of Applied Statistics} dependent on permission from VDGIF and some element of statistical novelty.
\end{document}




