\documentclass[11pt]{article}
\usepackage[pdftex]{color, graphicx}
\usepackage{amsmath, amsfonts, amssymb, mathrsfs}
\usepackage{dcolumn}
\usepackage{natbib}
\bibpunct{(}{)}{;}{a}{}{,}

\oddsidemargin=0.25in
\evensidemargin=0.25in
\textwidth=6in
\textheight=8.75in
\topmargin=-.5in
\footskip=0.5in

\title{Annotated Bibliography for HM Project}
\begin{document}

\maketitle

\section*{Survival Models}
\subsubsection*{Regression Models and Life-Tables}
This is the foundation for many survival models with censored data. Also includes a section on discrete survival analysis.
\begin{verbatim}
 @article{1972,
     jstor_articletype = {research-article},
     title = {Regression Models and Life-Tables},
     author = {Cox, D. R.},
     journal = {Journal of the Royal Statistical Society. Series B (Methodological)},
     jstor_issuetitle = {},
     volume = {34},
     number = {2},
     jstor_formatteddate = {1972},
     pages = {pp. 187-220},
     url = {http://www.jstor.org/stable/2985181},
     ISSN = {00359246},
     abstract = {The analysis of censored failure times is considered. It is assumed that on each individual are available values of one or more explanatory variables. The hazard function (age-specific failure rate) is taken to be a function of the explanatory variables and unknown regression coefficients multiplied by an arbitrary and unknown function of time. A conditional likelihood is obtained, leading to inferences about the unknown regression coefficients. Some generalizations are outlined.},
     language = {English},
     year = {1972},
     publisher = {Wiley for the Royal Statistical Society},
     copyright = {Copyright � 1972 Royal Statistical Society},
    }
\end{verbatim}
\section*{Censoring and Truncation}
\subsubsection*{Censoring and Truncation -- Highlighting the Differences}
The authors present Censoring and Truncation and clearly define the differences between the two.
\begin{verbatim}
@article{doi:10.1198/000313007X247049,
author = {Mandel, Micha},
title = {Censoring and Truncation--Highlighting the Differences},
journal = {The American Statistician},
volume = {61},
number = {4},
pages = {321-324},
year = {2007},
doi = {10.1198/000313007X247049},
URL = { http://dx.doi.org/10.1198/000313007X247049},
eprint = {http://dx.doi.org/10.1198/000313007X247049}
}
\end{verbatim}
\end{document}




