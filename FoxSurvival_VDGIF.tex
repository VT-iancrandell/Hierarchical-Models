\documentclass[aoas,preprint]{imsart}

\RequirePackage[OT1]{fontenc}
\RequirePackage{amsthm,amsmath}
\RequirePackage{natbib}
\RequirePackage[colorlinks,citecolor=blue,urlcolor=blue]{hyperref}
\usepackage{graphicx}

% settings
%\pubyear{2005}
%\volume{0}
%\issue{0}
%\firstpage{1}
%\lastpage{8}
\arxiv{arXiv:0000.0000}

\startlocaldefs
\numberwithin{equation}{section}
\theoremstyle{plain}
\newtheorem{thm}{Theorem}[section]
\endlocaldefs

\begin{document}
This article details an analysis conducted using fox enclosure data obtained from our collaboration with the Conservation Management Institute (CMI) at Virginia Tech. At the time, we (Andy Hoegh and Ian Crandell) were collaborators in the Laboratory for Interdisciplinary Statistical Analysis at Virginia Tech and assisted CMI researchers with the initial analysis of this data. While the attached article was prepared for a course project, we would be interested in continuing the work and submitting a research article. We do understand that this is a highly charged political issue and that public release of the information detailed here will require careful consideration. If preferred we would be willing to avoid disclosing the data or any identifying components. To avoid confusion, the analysis below is generated using a template for a journal but it has not been submitted.
 
 In regards to the analysis, there are a few assumptions made here that if incorrect may lead to different results. In particular we assume:
 \begin{enumerate}
 \item We are working with a complete dataset. This dataset contains information for 27 foxes. We seem to remember that there may be a more complete dataset, but we were not able to get our hands on it.
 \item We make an assumption on the number of dogs in the enclosure on `non-trial' days. Specifically we assume that it is not strictly zero, but varies between zero and about fifty. 
 \end{enumerate}
If either of these assumptions are off base, we'd be happy to rerun the results to determine what changes.

We didn't see the final results provided by CMI so it is hard to say if anything would change in this analysis. Nevertheless, we will summarize our findings here. The results of our analysis are fairly intuitive, in that as the number of dogs increases and the time a fox has spent in the enclosure decreases fox survival is expected to decrease. That is lots of dogs and foxes new to the enclosure is a recipe for high fox mortality rates. However, by placing a limit on the number of dogs or allowing foxes acclimation time without dogs, the daily survival rates are considerably higher.

We would be happy to answer any questions about our analysis and the results contained therein. Our emails can be found on the final page of the article.\\
\\
Sincerely,\\
\\
Andy Hoegh \& Ian Crandell
\newpage
\begin{frontmatter}
%\title{Assessing Fox Mortality during Fox Hunting Trials\thanksref{T1}}
\runtitle{Fox Mortality}
%\thankstext{T1}{Footnote to the title with the ``thankstext'' command.}

\begin{aug}
\author{\fnms{Andrew} \snm{Hoegh}\thanksref{m1}\ead[label=e1]{ahoegh@vt.edu}},
\author{\fnms{Ian} \snm{Crandell}\thanksref{m1}\ead[label=e2]{ian85@vt.edu}}


\thankstext{t1}{Some comment}
\runauthor{Hoegh \& Crandell}

\affiliation{Department of Statistics, Virginia Tech\thanksmark{m1}}

\address{Department of Statistics, Virginia Tech\\
Hutcheson Hall 401-D\\
Blacksburg, VA 24061\\
\printead{e1}\\
\phantom{E-mail:\ }\printead*{e2}}

\end{aug}

\begin{abstract}
Fox penning is a highly controversial practice whereby foxes are hunted with dogs in a large enclosed space. While there are commonalities with traditional horseback fox hunting, there are definite differences as well. The main difference is that with fox penning there is no chance for the fox to escape the enclosure, which violates the idea of fair chase. Using data from a fox enclosure in Virginia, we investigate the factors that influence a fox's chances to survive in such pens. We then use our model to examine possible changes to fox penning policy and what effect those changes may have on fox survival. We conclude that by either allowing the foxes to acclimate to their environment or by limiting the number of hunting dogs in the pen, we can improve fox survival.
\end{abstract}

\begin{keyword}
\kwd{Survival Analysis}
\kwd{Missing Data}
\kwd{Bayesian Modeling}
\end{keyword}

\end{frontmatter}

\section{Introduction}

Fox penning is a highly charged practice, both in Virginia and nationally. A major issue revolves around the concept of \emph{fair chase} \citep{posewitz} and more broadly of ethics of the hunter. Fair chase is defined as: `the ethical, sportsmanlike, and lawful pursuit and taking of any free-ranging wild, native North American big game animal in a manner that does not give the hunter an improper advantage over such animals' \citep{boone}. One could argue maintaining animals in enclosures for hunting or training purposes, cannot satisfy the requirements of fair chase.  However \cite{posewitz} claims fair chase ``is a balance that allows hunters to occasionally succeed, while animals generally avoid being taken.'' We focus on assessing fox mortality in fox pens. By identifying and controlling factors contributing to fox survival, we can help policy makers mimic the spirit of fair chase by enabling high survival rates of foxes.

As with any controversial practice, there are strong opinions on both sides of the issue. With this article we stick to objective, measurable characteristics like survival rate rather than delving into subjective assessments of ethics such as animal cruelty. With that in mind, we use data from a fox pen in Virginia to assess survival of foxes in enclosures. Specifically we focus on the effect of dogs in the pens and the time a fox has to acclimate to the new surroundings. Finally, a note of caution is necessary for these results. The data is not a product of a well designed experimental trial across several fox enclosures, but rather is collected at a single site. As such, caution should be exhibit when applying results from this pen to other enclosures, both in Virginia and nationally. Nevertheless, our assessment contains data driven results for evaluating policy implications for fox penning.

The remainder of this article follows as: Section 2 outlines the data in the study, Section 3 contains details on modeling, Section 4 describes the model results, Section 5 presents a set of hypothetical policy regimes, and Section 6 concludes with a discussion.

\section{Data} Upon being placed in the enclosure, some of the foxes were equipped with radio transmitters providing a way to determine the foxes survival during the study. This study contains information from 27 foxes over the course of 17 months from October 2002 to February 2004. During this time period there are two distinct types of days: trial days and non-trial days. Trial days are typically organized by hunt clubs and consist of competitions where groups of owners bring their dogs to the enclosures. On trial days there is a record of the number of dogs in the pen is available. The number of dogs ranges from 50 to 752 during the trials. On non-trial days the number of dogs is not available. Unfortunately, this does not imply that no dogs are present. Rather, the number of dogs typically is quite small and tends to be less than fifty.

The foxes are not all placed into the pen at the start of the study period, instead they are placed into the enclosure throughout the study period. Of the 27 foxes, 6 survive to the end of the study. Figure \ref{fig:SurvTime} contains a box plot along with points for each fox of the survival time for the foxes that perished during the study period. 
\begin{figure}[h!]
\begin{center}
\caption{Survival Time of Foxes, dots denote time in enclosure at death and 'x' denotes time in enclosure for foxes surviving the study period}
\includegraphics[width=.85\textwidth]{Survival_Time.pdf}
\label{fig:SurvTime}
\end{center}
\end{figure}
Additionally this figure also displays the time in the enclosure for the foxes that survived to the end of the study. For policy considerations, we consider the number of dogs in the enclosure and the time a fox has had to acclimated to the enclosure as the factors for fox survival. Obviously more dogs hunting the foxes should lower the survival rate. Acclimation time allows a fox to establish a territory and become acquainted with the area and perhaps find dens or other features that increase survival probabilities.

\subsection{Missing Data}
As the number of dogs in the enclosure is unknown and not strictly zero, our models need to account for this fact. Simply using zeros (or any other constant) would fail to capture the uncertainty present in the data and lead to flawed inferences. Generally speaking this is known as missing data \citep{little}. A common classical technique for dealing with missing data is multiple imputation \citep{rubin1976,rubin}. In essence, imputation is conducted several times to consider the uncertainty inherent in predicting unknown values.

From a Bayesian perspective, inferences are made from the posterior distribution, $P(\Theta|X,Y).$ With missing covariate data consider $
	\mathcal{X}=\left[
	\begin{array}{ll}
	X \\
	X^{*} 
	\end{array}
	\right]$ \\ 
	and $X^{*}$ denotes missing covariates. Then given a prior distribution on the missing covariates, uncertainty in the missing data is accounted for in a typical Bayesian framework by integrating over the distribution \citep{boone2009}. Specifically, the posterior distribution can now be computed as $$P(\Theta|X,Y), = \int P(\Theta|X,X^{*},Y) p(X^{*}) dX^{*}.$$
Thus, multiple imputation is a classical technique that mimics Bayesian methods for missing data, which we implement in this study.
\section{Model} While traditional survival models such as the Cox proportional hazards model \citep{cox} or a discrete survival model were considered, this scenario does not fit neatly into the survival framework \citep{klein}. This is because we have covariate information on days, the number of dogs, where in the Cox model, all covariate data pertains to the observational unit, the foxes in our case. A complicating factors is that the age of the foxes is unknown when they are placed into the pen and can be expected to vary considerably from fox to fox. Furthermore, analogous to a survival model our ``treatment'' is the number of dogs. However, the effect of dogs would be confined to a single day. In other words, dogs in the pen on day $t$ would not lead to higher mortality on day $t+1$. Finally, we are concerned with policy implications for survival of foxes in enclosures and it is more practical to impose restrictions on daily behaviors than to consider the lifetime of a fox. Given these two considerations, we use a model for the survival of fox on a given day based on the number of dogs in the enclosure and the acclimation time of the fox.

Specifically a binary regression framework is invoked using a generalized linear model with a probit link function.
\begin{eqnarray}
y_{it} &\sim& Bernoulli(p_{it}) \\
probit(p_{it}) & = & \alpha + \mathcal{X}_{it}\boldsymbol{\beta} + \theta_{i} \\
\theta_{i} &\sim& N(0, \phi^{-1})
\end{eqnarray}
where $i = \{1,...27\} $(fox), and $t= \{1,...,T\}$ (time). The variable $y_{it}$ is a binary variable corresponding to survival($y_{it}=1$) or death of fox $i$ on day $t$. The matrix $\mathcal{X}_{it} = [Dogs_t\; log.exp_{it}\; (Dogs*log.exp)_{it}],$ contains the number of dogs on day $t$, the log of experience for fox $i$ on day $t$ and their interaction. Furthermore, in line with previous notation $\mathcal{X}$ includes the missing data components for the number of dogs. The $\theta_i$ terms are random effects for each fox. Let $R_{it}$ be the risk matrix, where
\[
    R_{it}=\left\{
                \begin{array}{ll}
                  1 \quad \text{ if fox i is alive and collared on day t-1}\\
                  0 \quad \text{otherwise}
                \end{array}
              \right.
  \]
  For calculation of the likelihood, only elements where $R_{it}=1$ are considered. As in \cite{albert}, we use data augmentation where 
  \begin{eqnarray}
  Z_{it} \sim N( \alpha + X_{it}\boldsymbol{\beta} +\theta_{i})
  \end{eqnarray}
  and
\[
    Y_{it}=\left\{
                \begin{array}{ll}
                  1 \quad  Z_{it} > 0\\
                  0 \quad Z_{it} \leq 0
                \end{array}
              \right.
  \]
  For prior specification, we use conjugate priors: $p(\alpha) \sim N(\alpha;0,1), \boldsymbol{\beta} \sim N(\boldsymbol{\beta};0,\Sigma),$ where $\Sigma$ is a diagonal matrix scaled such that the diagonal elements, $\sigma_{ii},$ are twice the range of the data. An informative prior $Gamma(5,5)$ is placed on the precision for the random effects, $\phi$. This insures that the fox random effects are well behaved and do not drift off to $\infty$ as might be expected for foxes that survive all of their trials (particularly under a maximum likelihood paradigm). This analysis also requires priors on the missing data, the number of dogs on non-trial days. Given we believe that the number of dogs is generally less than 50 based on expert opinion, a truncated normal prior is used. Specifically $p(X^{*}) \sim N(X^{*}; 0 ,10^2,0,50)$. By considering dogs per acre, a one-to-one transformation from number of dogs, the use of a continuous truncated normal prior is defensible. 
  
\section{Results}
The use conjugate priors enables Gibbs sampling for each of the parameters, including the missing data. The sampler was run for 500,000 iterations to insure convergence for all of the parameters. The large number of iterations were necessitated for proper mixing, mostly a result of the missing data. Estimates of the coefficients and 95\% credible intervals are give in Table \ref{tab:PostEst}. The random effects parameter $\phi$ denotes (unsurprisingly) that different foxes have different survival probabilities. Furthermore, inclusion of the random effects controls for variation amongst the foxes and allows valid inferences within our model framework.
\begin{table}[h!]
	\begin{center}
		\begin{tabular}{|c|c|c|}
			\hline
			& mean & CI \\
			\hline
			$\alpha$ & 5.4 & (3.0,11.6) \\
			$\beta_{dogs}$ & -.008 & (-.015, -.004) \\
			$\beta_{exp}$ & -.65 & (-2.04,-.10) \\
			$\beta_{dogs*exp}$ & .0014 & (.0006, .0030)\\
			$ \phi$ & 0.79 & (0.07, 2.06) \\
			\hline
		\end{tabular}
		\label{tab:PostEst}
	\end{center}
	\caption{Table of model coefficients.}
\end{table}
 This shows that no credible interval contains 0, indicating that all predictors are useful. Interpretation of the main effects of the individual coefficients is complicated by the presence of the interaction term, but we can understand the model output by examining the heat map in Figure \ref{fig:SurvProb2}. 
\begin{figure}[h!]
\begin{center}
\caption{Survival Probability of Foxes as a function of experience and dogs in enclosure.}
\includegraphics[width=1\textwidth]{survivalprob.pdf}
\label{fig:SurvProb2}
\end{center}
\end{figure}
We see that fox survival is lowest when there are many dogs and inexperienced foxes. This survival rises drastically as either the number of dogs drops or the experience of the foxes increases.

\section{Policy Analysis}
One of the motivating goals of this paper was to find ways to make fox penning less cruel. To learn about this, we considered how fox survival might change if the pens were to make policy changes. We considered four different situations:
	\begin{itemize}
		\item Regime A: No constraints on number of dogs or allotted acclimation time. This is currently in place.
		\item Regime B: Two weeks acclimation time with no dogs.
		\item Regime C: No more than 400 dogs allowed in pen at a time.
		\item Regime D: Two weeks acclimation time and 400 dog limit.
	\end{itemize}
Given these four regimes we explored the expected survival for a typical fox if it had been placed in the enclosure on October 1, 2002, the first day of the study period. For regimes (B-D) we zero out the number of dogs for the acclimation time and restrict the number of dogs to the specified limit on days where the actual number of dogs exceeded the limit.
These four regimes produced the curves in Figure \ref{fgr:1}, where the solid line represents the mean expected survival probability up to dat $t$ and the dashed lines denote Bayesian credible intervals.
\begin{figure}[h]
	\begin{center}$
		\begin{array}{cc}
		\includegraphics[width=.5\textwidth]{RegA.pdf} & \includegraphics[width=.5\textwidth]{RegB.pdf}\\
		\includegraphics[width=.5\textwidth]{RegC.pdf} & \includegraphics[width=.5\textwidth]{RegD.pdf}
		\end{array}$
	\end{center}
	\caption{Survival Probabilities for the four regimes}
	\label{fgr:1}
\end{figure}
These are not survival curves, per se, but rather the probability of a fox surviving until day $t$ given the trials conducted in the enclosure. Again, the survival rates are dependent of the number of dogs in the enclosure. We use the reported values from the trials coupled with our missing value imputation subject to restrictions for a given regime. 

We see a precipitous drop in fox survival for regime A but not for any of the other three regimes. In fact, the other three plots look very similar. This accords with the heat map of survival probabilities, where we see a significant increase in survival probabilities as either the fox becomes more acclimated or the number of dogs is decreased. 

\subsection{Study Limitations}
Due to limitations in the data, it is not possible to answer some relevant questions. For instance, we cannot do a comparison of survival rates for penned versus wild foxes. In the wild, foxes generally live around five years \citep{hunter}, but we cannot compare this to penned foxes as the ages of the foxes place in the enclosure are unknown. Second, it is likely that individual characteristics of the pen contribute to fox survival and interact with other covariates. This is not possible to measure since the data come from a single pen. Hence, caution is warranted in applying these results to other fox pens. Nevertheless, we encourage the use of any and all data for policy making; however, limitations need to be understood. With regard to fox acclimation time, scientific expertise is necessary to understand the implications. For instance, does additional time provide opportunity for the foxes to learn the lay of the land and find hiding places and furthermore how is this impacted by territories claimed by other foxes. A lurking variable may be the total number of foxes in the pen, which unfortunately is not available. It may be the case that a pen contains ample refuge for a given number of foxes, but beyond that number fox fatalities will be expected to increase.

Given the limitations of this study an ideal case would provide the opportunity to perform a more rigorous study in the future. It would be fruitful to examine data from other Virginia fox pens. We would examine terrain features to learn about their effects on mortality, study how those features interact with fox experience and dog count, and perform cross validation of our models on data from some of the other pens.

\section{Discussion}
While our analysis has been about fox survival, we do not take a stand on issues of cruelty or attempt to determine an \emph{acceptable} survival rate. Our focus is strictly on factors affecting survival and we present the inferences drawn from the data. The issues of cruelty and survival are best left in the hands of policy makers.

We find that significant fox experience and the number of dogs as well as their interaction to significantly influence fox survival probabilities. In particular low levels of fox acclimation and high number of dogs are problematic. Our regime analyses reflect this fact, as fewer than half of the foxes survive beyond the first couple of days day under the current policy. Fortunately, it only takes a small change in policy to greatly improve fox survival. By either allowing the foxes acclimation time or by limiting the number of dogs in the pen we can eliminate the drastic fox mortality we see in the current regime and balance the hunters success with the foxes ability to escape.

For our future work on this project, there are a few additional considerations. First, we'd like to get a better sense of the number of dogs in the pen on non-trial days. This would likely consist of a mixture distribution of a point mass at zero and perhaps a negative binomial distribution. While this would require a Metropolis-Hastings sampler, it may be a more sound prior than our truncated normal. We'd like to do model selection in a Bayesian framework, comparing models with and without the interaction term as well as the use of log transformed variables for both number of dogs and experience.
\bibliographystyle{asa}
\bibliography{refsFox}

\end{document}
