\documentclass{beamer}

\usepackage{pgf,pgfpages}
\usepackage{graphicx}
\usepackage{units}
\usepackage[utf8]{inputenc}
\usepackage{lmodern}
\usepackage{subfig}
\makeatletter
\newcommand\ChangeItemFont[3]{%
\renewcommand{\itemize}[1][]{%
  \beamer@ifempty{##1}{}{\def\beamer@defaultospec{#1}}%
  \ifnum \@itemdepth >2\relax\@toodeep\else
    \advance\@itemdepth\@ne
    \beamer@computepref\@itemdepth% sets \beameritemnestingprefix
    \usebeamerfont{itemize/enumerate \beameritemnestingprefix body}%
    \usebeamercolor[fg]{itemize/enumerate \beameritemnestingprefix body}%
    \usebeamertemplate{itemize/enumerate \beameritemnestingprefix body begin}%
    \list
      {\usebeamertemplate{itemize \beameritemnestingprefix item}}
      {\def\makelabel####1{%
          {%
            \hss\llap{{%
                \usebeamerfont*{itemize \beameritemnestingprefix item}%
                \usebeamercolor[fg]{itemize \beameritemnestingprefix item}####1}}%
          }%
        }%
  \ifnum\@itemdepth=1\relax
    #1%
  \else
  \ifnum\@itemdepth=2\relax
    #2%
  \else
  \ifnum\@itemdepth=3\relax
    #3%
  \fi%
  \fi%
  \fi%
  }
  \fi%
  \beamer@cramped%
  \raggedright%
  \beamer@firstlineitemizeunskip%
}}
\makeatother

\mode<presentation>
{
  \usetheme{ift}
  \setbeamercovered{transparent}
  \setbeamertemplate{items}[square]
}

\usefonttheme[onlymath]{serif}
\setbeamerfont{frametitle}{size=\LARGE,series=\bfseries}

\definecolor{uibred}{RGB}{102,0, 0}
\definecolor{uibblue}{RGB}{102,0, 0}
\definecolor{uibgreen}{RGB}{102,0, 0}
%\definecolor{uibgreen}{RGB}{50, 105, 0}
\definecolor{uiborange}{RGB}{102,0, 0}


\beamertemplatenavigationsymbolsempty



\defbeamertemplate{enumerate item}{mycircle}
{
  %\usebeamerfont*{item projected}%
  %\usebeamercolor[bg]{item projected}%
  \begin{pgfpicture}{0ex}{0ex}{1.5ex}{0ex}
	%\pgfcircle[fill]{\pgfpoint{0pt}{.75ex}}{1.25ex}
    \pgfbox[center,base]{\color{uibblue}\insertenumlabel.}
  \end{pgfpicture}%
}
[action]
{\setbeamerfont{item projected}{size=\scriptsize}}
\setbeamertemplate{enumerate item}[mycircle]




\title{Assessing Fox Mortality in Fox Penning Trials}
\author{Ian Crandell \& Andy Hoegh}
\institute{
Department of Statistics, Virginia Tech
}
\date{11.4.2014}

\begin{document}

\newcommand\Fontvi{\fontsize{6}{7.2}\selectfont}

\setbeamertemplate{background}
 {\includegraphics[width=\paperwidth,height=\paperheight]{frontpage}}
\setbeamertemplate{footline}[default]

\begin{frame}
  \titlepage
  \vspace{5cm}
\end{frame}

%
% Set the background for the rest of the slides.
% Insert infoline at the end
%
\setbeamertemplate{background}
 {\includegraphics[width=\paperwidth,height=\paperheight]{slide_small}}
\setbeamertemplate{footline}[ifttheme]

%%%%%%%%%%%%%%%%%%%%%%%%%%%%%%%%%%%%%%%%%%%%%%%%%%%%%%
%%%%%%%%%%%%%%%%%%%%%%%%%%%%%%%%%%%%%%%%%%%%%%%%%%%%%%
\section{Motivation}
\subsection{Intro}
\begin{frame}
	\frametitle{What is fox penning?}
	Add a picture of traditional horseback hunting \\
	ideas of fair chase\\
	difference in hunting and penning\\
	
%	\frametitle{Goal: Predict Civil Unrest}
%	\begin{columns}
%		\column{0.5\textwidth}
%			\parbox[c][0.9\textheight]{0.9\textwidth}
%				{\includegraphics<1>[width=\textwidth]{Venezuela.png}}
%		\column{0.5\textwidth}
%			\parbox[c][0.9\textheight]{0.9\textwidth}
%			{\includegraphics<1>[width=\textwidth]{BrazilProtestjpg.jpg}}
%	\end{columns}
\end{frame}
%%%%%%%%%%%%%%%%%%%%%%%%%%%%%%%%%%%%%%%%%%%%%%%%%%%%%%
\begin{frame}
	\frametitle{Origins of the Project}
	Talk about LISA \\
	CMI \\
	VDGIF\\
	Relevance - of project	
\end{frame}
%%%%%%%%%%%%%%%%%%%%%%%%%%%%%%%%%%%%%%%%%%%%%%%%%%%%%%
\begin{frame}
	\frametitle{Policy Question}
	How to minimize risk to fox. This is not a question of no fatalities, but what actions could reduce risk.
\end{frame}
%%%%%%%%%%%%%%%%%%%%%%%%%%%%%%%%%%%%%%%%%%%%%%%%%%%%%%
%%%%%%%%%%%%%%%%%%%%%%%%%%%%%%%%%%%%%%%%%%%%%%%%%%%%%%
\section{Data Overview}
\begin{frame}
	\frametitle{Data Overview}
	what data is available \\
	segue into missing data
	\end{frame}
%%%%%%%%%%%%%%%%%%%%%%%%%%%%%%%%%%%%%%%%%%%%%%%%%%%%%

\begin{frame}
	\frametitle{Missing Data}
	quick overview of missing data \\
	add math to show Data Augmentation to integrate out
	\end{frame}
%%%%%%%%%%%%%%%%%%%%%%%%%%%%%%%%%%%%%%%%%%%%%%%%%%%%%%
%%%%%%%%%%%%%%%%%%%%%%%%%%%%%%%%%%%%%%%%%%%%%%%%%%%%%%
\section{Modeling}
\subsection{Modeling Specification}
\begin{frame}
	\frametitle{Modeling Specification}
\begin{eqnarray}
y_{it} &\sim& Bernoulli(p_{it}) \\
probit(p_{it}) & = & \alpha + X_{it}\beta+ \theta_{it} \\
\theta_{it} &\sim& N(0, \phi^{-1})
\end{eqnarray}
where $X_{it} = [Num.dogs_t\; log.experience_{it}\; (num.dogs*log.experience)_{it}],$ $i = \{1,...27\} $(fox), and $t= \{1,...,T\}$ (trials). Let $R_{it}$ be the risk matrix, where
\[
    R_{it}=\left\{
                \begin{array}{ll}
                  1 \quad \text{ if fox i is alive and collared on day t-1}\\
                  0 \quad \text{otherwise}
                \end{array}
              \right.
  \]
 \end{frame}
%%%%%%%%%%%%%%%%%%%%%%%%%%%%%%%%%%%%%%%%%%%%%%%%%%%%%%
\begin{frame}
	\frametitle{Modeling Specification}
	 We use data augmentation where 
	 \[
    Y_{it}=\left\{
                \begin{array}{ll}
                  1 \quad  Z_{it} > 0\\
                  0 \quad Z_{it} \leq 0
                \end{array}
              \right.
  \]
and then
  \begin{eqnarray}
  Z_{it} \sim N( \alpha + X_{it}\beta +\theta_{it},1).
  \end{eqnarray}
\\
	discuss priors here and missing data piece
\end{frame}
%%%%%%%%%%%%%%%%%%%%%%%%%%%%%%%%%%%%%%%%%%%%%%%%%%%%%%
%%%%%%%%%%%%%%%%%%%%%%%%%%%%%%%%%%%%%%%%%%%%%%%%%%%%%%
\section{Results}
\subsection{??}
\begin{frame}
	\frametitle{Posterior Summaries}
\begin{table}[h]
%\caption{default}
\begin{center}
\begin{tabular}{|c|c|c|}
\hline
 & mean & CI \\
 \hline
  $\alpha$ & 5.4 & (3.0,11.6) \\
 $\beta_{dogs}$ & -.008 & (-.015, -.004) \\
 $\beta_{exp}$ & -.65 & (-2.04,-.10) \\
 $\beta_{dogs*exp}$ & .0014 & (.0006, .0030)\\
 \hline
\end{tabular}
\end{center}
\label{default}
\end{table}%

\end{frame}
%%%%%%%%%%%%%%%%%%%%%%%%%%%%%%%%%%%%%%%%%%%%%%%%%%%%%%
\subsection{??}
\begin{frame}
	\frametitle{Response Surface}
	\begin{columns}
		\column{0.5\textwidth}
			\parbox[c][0.9\textheight]{0.9\textwidth}
				{\includegraphics<1>[width=\textwidth]{ResponseSurface.pdf}}
		\column{0.5\textwidth}
			\parbox[c][0.9\textheight]{0.9\textwidth}
			{\includegraphics<1>[width=\textwidth]{SurvivalProb.pdf}}
	\end{columns}

\end{frame}
%%%%%%%%%%%%%%%%%%%%%%%%%%%%%%%%%%%%%%%%%%%%%%%%%%%%%%
\subsection{??}
\begin{frame}
	\frametitle{Regime Study}
	All regimes assume a fox was placed in the pen on the first day of the study.
	\begin{itemize}
	\item Regime A: No constraints on number of dogs or allotted acclimation time.
	\item Regime B: Two weeks acclimation time with no dogs.
	\item Regime C: No more than 400 dogs allowed in pen at a time.
	\item Regime D: Two weeks acclimation time and 400 dog limit.
	\end{itemize}
%	At the finest level $p(\boldsymbol{Y}_{tL_j}|\boldsymbol{\mu}_{tL_j}) \overset{ind}{\sim} \text{Poisson}
\end{frame}

\subsection{??}
\begin{frame}
	\frametitle{Regime Survival Plots}
	\begin{columns}
		\column{0.5\textwidth}
			\parbox[c][0.9\textheight]{0.9\textwidth}
				{\includegraphics<1>[width=\textwidth]{RegA.pdf}}
						\column{0.5\textwidth}
			\parbox[c][0.9\textheight]{0.9\textwidth}
			{\includegraphics<1>[width=\textwidth]{RegB.pdf}}
	\end{columns}
\end{frame}
\subsection{??}
\begin{frame}
	\frametitle{Regime Survival Plots}
	\begin{columns}
		\column{0.5\textwidth}
			\parbox[c][0.9\textheight]{0.9\textwidth}
				{\includegraphics<1>[width=\textwidth]{RegC.pdf}}
						\column{0.5\textwidth}
			\parbox[c][0.9\textheight]{0.9\textwidth}
			{\includegraphics<1>[width=\textwidth]{RegD.pdf}}
	\end{columns}
\end{frame}


\subsection{Final Thoughts}
\begin{frame}
	\frametitle{What's Ahead}

\end{frame}

\end{document}

