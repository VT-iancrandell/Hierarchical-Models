\documentclass{beamer}

\usepackage{pgf,pgfpages}
\usepackage{graphicx}
\usepackage{units}
\usepackage[utf8]{inputenc}
\usepackage{lmodern}
\usepackage{subfig}
\makeatletter
\newcommand\ChangeItemFont[3]{%
\renewcommand{\itemize}[1][]{%
  \beamer@ifempty{##1}{}{\def\beamer@defaultospec{#1}}%
  \ifnum \@itemdepth >2\relax\@toodeep\else
    \advance\@itemdepth\@ne
    \beamer@computepref\@itemdepth% sets \beameritemnestingprefix
    \usebeamerfont{itemize/enumerate \beameritemnestingprefix body}%
    \usebeamercolor[fg]{itemize/enumerate \beameritemnestingprefix body}%
    \usebeamertemplate{itemize/enumerate \beameritemnestingprefix body begin}%
    \list
      {\usebeamertemplate{itemize \beameritemnestingprefix item}}
      {\def\makelabel####1{%
          {%
            \hss\llap{{%
                \usebeamerfont*{itemize \beameritemnestingprefix item}%
                \usebeamercolor[fg]{itemize \beameritemnestingprefix item}####1}}%
          }%
        }%
  \ifnum\@itemdepth=1\relax
    #1%
  \else
  \ifnum\@itemdepth=2\relax
    #2%
  \else
  \ifnum\@itemdepth=3\relax
    #3%
  \fi%
  \fi%
  \fi%
  }
  \fi%
  \beamer@cramped%
  \raggedright%
  \beamer@firstlineitemizeunskip%
}}
\makeatother

\mode<presentation>
{
  \usetheme{ift}
  \setbeamercovered{transparent}
  \setbeamertemplate{items}[square]
}

\usefonttheme[onlymath]{serif}
\setbeamerfont{frametitle}{size=\LARGE,series=\bfseries}

\definecolor{uibred}{RGB}{102,0, 0}
\definecolor{uibblue}{RGB}{102,0, 0}
\definecolor{uibgreen}{RGB}{102,0, 0}
%\definecolor{uibgreen}{RGB}{50, 105, 0}
\definecolor{uiborange}{RGB}{102,0, 0}


\beamertemplatenavigationsymbolsempty



\defbeamertemplate{enumerate item}{mycircle}
{
  %\usebeamerfont*{item projected}%
  %\usebeamercolor[bg]{item projected}%
  \begin{pgfpicture}{0ex}{0ex}{1.5ex}{0ex}
	%\pgfcircle[fill]{\pgfpoint{0pt}{.75ex}}{1.25ex}
    \pgfbox[center,base]{\color{uibblue}\insertenumlabel.}
  \end{pgfpicture}%
}
[action]
{\setbeamerfont{item projected}{size=\scriptsize}}
\setbeamertemplate{enumerate item}[mycircle]




\title{Predicting Civil Unrest: \\Venturing Through Latin America}
\author{Andy Hoegh, Scotland Leman, Marco Ferreira}
\institute{
Department of Statistics, Virginia Tech
}
\date{8.3.2014}

\begin{document}

\newcommand\Fontvi{\fontsize{6}{7.2}\selectfont}

\setbeamertemplate{background}
 {\includegraphics[width=\paperwidth,height=\paperheight]{frontpage}}
\setbeamertemplate{footline}[default]

\begin{frame}
  \titlepage
  \vspace{5cm}
\end{frame}

%
% Set the background for the rest of the slides.
% Insert infoline at the end
%
\setbeamertemplate{background}
 {\includegraphics[width=\paperwidth,height=\paperheight]{slide_small}}
\setbeamertemplate{footline}[ifttheme]

%%%%%%%%%%%%%%%%%%%%%%%%%%%%%%%%%%%%%%%%%%%%%%%%%%%%%%
%%%%%%%%%%%%%%%%%%%%%%%%%%%%%%%%%%%%%%%%%%%%%%%%%%%%%%
\section{Motivation}
\subsection{Goal}
\begin{frame}

%	\frametitle{Goal: Predict Civil Unrest}
%	\begin{columns}
%		\column{0.5\textwidth}
%			\parbox[c][0.9\textheight]{0.9\textwidth}
%				{\includegraphics<1>[width=\textwidth]{Venezuela.png}}
%		\column{0.5\textwidth}
%			\parbox[c][0.9\textheight]{0.9\textwidth}
%			{\includegraphics<1>[width=\textwidth]{BrazilProtestjpg.jpg}}
%	\end{columns}
\end{frame}
%%%%%%%%%%%%%%%%%%%%%%%%%%%%%%%%%%%%%%%%%%%%%%%%%%%%%%
\subsection{Data Overview}
\begin{frame}
	\frametitle{Data Overview}
	Data is collected from ten countries. Each instance of civil unrest contains four elements: \{location, date, event type,
	population\}. For instance, 
	\begin{eqnarray*}
	\mathcal{E} &=& \{Mexico,Quintana\; Roo / 06.27.2014 / \\
	&&Employment \;\& \;Wages - not\; violent / medical \} 
	\end{eqnarray*}
	\end{frame}
%%%%%%%%%%%%%%%%%%%%%%%%%%%%%%%%%%%%%%%%%%%%%%%%%%%%%%	
\subsection{Data Visualization}
\begin{frame}
%	\frametitle{Data Visualization: Spatial}
%	\includegraphics[width=1\textwidth]{Spatial_Counts.png}
	\end{frame}
%%%%%%%%%%%%%%%%%%%%%%%%%%%%%%%%%%%%%%%%%%%%%%%%%%%%%%
\subsection{Research Question}
\begin{frame}
	\frametitle{Statistical Research Question}
 	{\bf Motivating Question:} How does an observed event effect expected future predictive intensities?
 	\\~\\
	{\bf Challenge:} Incorporating structure to account for spatial, temporal effects as well as 
	cross-correlation between event types and populations.
\end{frame}
%%%%%%%%%%%%%%%%%%%%%%%%%%%%%%%%%%%%%%%%%%%%%%%%%%%%%%
%%%%%%%%%%%%%%%%%%%%%%%%%%%%%%%%%%%%%%%%%%%%%%%%%%%%%%
\section{Modeling}
\subsection{Modeling Options}
\begin{frame}
	\frametitle{Modeling Options}
	\begin{enumerate}
	\item Independent state space model at each state.
	\begin{itemize}
	\item Pros: Easy parallel computation
	\item Cons: Lacks any spatial structure / sharing across states
	\end{itemize}
	\item Hierarchical random effects model with CAR
	\begin{itemize}
	\item Pros: Accounts for spatial structure
	\item Cons: Computationally demanding
	\end{itemize}
	\end{enumerate}
	or \\~\\
	use multiscale modeling.
\end{frame}
%%%%%%%%%%%%%%%%%%%%%%%%%%%%%%%%%%%%%%%%%%%%%%%%%%%%%%
\subsection{Multiscale Visualization}
\begin{frame}
	\frametitle{Multiscale Visualization}
%	\begin{figure}[h!]
%	\centering
%	\includegraphics[width=1\textwidth]{Multiscale.pdf}
%	\end{figure}
\end{frame}
%%%%%%%%%%%%%%%%%%%%%%%%%%%%%%%%%%%%%%%%%%%%%%%%%%%%%%
%%%%%%%%%%%%%%%%%%%%%%%%%%%%%%%%%%%%%%%%%%%%%%%%%%%%%%
\section{Multiscale Factorization}
\subsection{Multivariate Poisson}
\begin{frame}
	\frametitle{Multiscale Factorization}
%	At the finest level $p(\boldsymbol{Y}_{tL_j}|\boldsymbol{\mu}_{tL_j}) \overset{ind}{\sim} \text{Poisson}
%	(\boldsymbol{\mu}_{tL_j})$ then:
	\small
	\begin{eqnarray*}
	\prod_{j=1}^{n_L} p(\boldsymbol{Y}_{tL_j}|\boldsymbol{\mu}_{tL_j})= \prod_{j=1}^{n_1} p(\boldsymbol{Y}_{t1_j}
	|\boldsymbol{\mu}_{t1_j}) \prod_{i=1}^I\prod_{l=1}^{L-1} \prod_{j=1}^{n_l} 
	p(\boldsymbol{Y}_{t{D_{l_j}i}}|Y_{tl_ji},\boldsymbol{\omega}_{tl_ji}),
	\end{eqnarray*}
	where $\boldsymbol{Y}_{t1_j} = \{Y_{t1_j1}, ..., Y_{t1_jI}\}$ is a vector of multivariate responses at 
	time $t$ for the first (coarse) level. 
%	At the coarse level $ p(\boldsymbol{Y}_{t1_j}|\boldsymbol{\mu}_{t1_j}) \overset{ind}{\sim} \text{Poisson}
%	(\boldsymbol{\mu}_{t1_j})$ and at the finer levels of the hierarchy 
%	$p(\boldsymbol{Y}_{t{D_{l_j}i}}|Y_{tl_ji},\boldsymbol{\omega}_{tl_ji}) \sim \text{Multinomial}(Y_{tl_ji},
%	\boldsymbol{\omega}_{tl_ji}).$ 
%	The vector $\boldsymbol{\omega}_{tl_ji}$ contains the multiscale coefficients for the descendants of region 
%	($l_j$) for variable $i$ at time $t$. 
\end{frame}
%%%%%%%%%%%%%%%%%%%%%%%%%%%%%%%%%%%%%%%%%%%%%%%%%%%%%%


\subsection{Final Thoughts}
\begin{frame}
	\frametitle{What's Ahead}
	\begin{itemize}
	\item Incorporating covariates: specifically our group has a set of algorithms that mine social media to predict
	 future protests.
	\item Choosing/Learning the multiscale structure.
	\end{itemize}
\end{frame}

\end{document}

