\documentclass{beamer}

\usepackage{pgf,pgfpages}
\usepackage{graphicx}
\usepackage{units}
\usepackage[utf8]{inputenc}
\usepackage{lmodern}
\usepackage{subfig}
\makeatletter
\newcommand\ChangeItemFont[3]{%
\renewcommand{\itemize}[1][]{%
  \beamer@ifempty{##1}{}{\def\beamer@defaultospec{#1}}%
  \ifnum \@itemdepth >2\relax\@toodeep\else
    \advance\@itemdepth\@ne
    \beamer@computepref\@itemdepth% sets \beameritemnestingprefix
    \usebeamerfont{itemize/enumerate \beameritemnestingprefix body}%
    \usebeamercolor[fg]{itemize/enumerate \beameritemnestingprefix body}%
    \usebeamertemplate{itemize/enumerate \beameritemnestingprefix body begin}%
    \list
      {\usebeamertemplate{itemize \beameritemnestingprefix item}}
      {\def\makelabel####1{%
          {%
            \hss\llap{{%
                \usebeamerfont*{itemize \beameritemnestingprefix item}%
                \usebeamercolor[fg]{itemize \beameritemnestingprefix item}####1}}%
          }%
        }%
  \ifnum\@itemdepth=1\relax
    #1%
  \else
  \ifnum\@itemdepth=2\relax
    #2%
  \else
  \ifnum\@itemdepth=3\relax
    #3%
  \fi%
  \fi%
  \fi%
  }
  \fi%
  \beamer@cramped%
  \raggedright%
  \beamer@firstlineitemizeunskip%
}}
\makeatother

\mode<presentation>
{
  \usetheme{ift}
  \setbeamercovered{transparent}
  \setbeamertemplate{items}[square]
}

\usefonttheme[onlymath]{serif}
\setbeamerfont{frametitle}{size=\LARGE,series=\bfseries}

\definecolor{uibred}{RGB}{102,0, 0}
\definecolor{uibblue}{RGB}{102,0, 0}
\definecolor{uibgreen}{RGB}{102,0, 0}
%\definecolor{uibgreen}{RGB}{50, 105, 0}
\definecolor{uiborange}{RGB}{102,0, 0}


\beamertemplatenavigationsymbolsempty



\defbeamertemplate{enumerate item}{mycircle}
{
  %\usebeamerfont*{item projected}%
  %\usebeamercolor[bg]{item projected}%
  \begin{pgfpicture}{0ex}{0ex}{1.5ex}{0ex}
	%\pgfcircle[fill]{\pgfpoint{0pt}{.75ex}}{1.25ex}
    \pgfbox[center,base]{\color{uibblue}\insertenumlabel.}
  \end{pgfpicture}%
}
[action]
{\setbeamerfont{item projected}{size=\scriptsize}}
\setbeamertemplate{enumerate item}[mycircle]




\title{Assessing Fox Mortality in Fox Penning Trials}
\author{Ian Crandell \& Andy Hoegh}
\institute{
Department of Statistics, Virginia Tech
}
\date{11.4.2014}

\begin{document}

\newcommand\Fontvi{\fontsize{6}{7.2}\selectfont}

\setbeamertemplate{background}
 {\includegraphics[width=\paperwidth,height=\paperheight]{frontpage}}
\setbeamertemplate{footline}[default]

\begin{frame}
  \titlepage
  \vspace{5cm}
\end{frame}

%
% Set the background for the rest of the slides.
% Insert infoline at the end
%
\setbeamertemplate{background}
 {\includegraphics[width=\paperwidth,height=\paperheight]{slide_small}}
\setbeamertemplate{footline}[ifttheme]

%%%%%%%%%%%%%%%%%%%%%%%%%%%%%%%%%%%%%%%%%%%%%%%%%%%%%%
%%%%%%%%%%%%%%%%%%%%%%%%%%%%%%%%%%%%%%%%%%%%%%%%%%%%%%
\section{Motivation}
\subsection{Intro}
\begin{frame}
	\frametitle{What is fox penning?}
	\begin{columns}
		\column{0.5\textwidth}
		\parbox[c][0.9\textheight]{0.9\textwidth}
		{\includegraphics<1>[width=\textwidth]{foxhunting.jpg}}
		\column{0.5\textwidth}
		\parbox[c][0.9\textheight]{0.9\textwidth}
		{\begin{itemize}
				\item Fox hunting is an ancient sport where human trainers and dogs hunt wild foxes.
				\item For fox penning, the foxes are brought into an enclosure and are then hunted for training or entertainment.
				\item Legislation restricting the growth of the practice has recently been passed in Virginia.
			\end{itemize}}
	\end{columns}
%	\frametitle{Goal: Predict Civil Unrest}
%	\begin{columns}
%		\column{0.5\textwidth}
%			\parbox[c][0.9\textheight]{0.9\textwidth}
%				{\includegraphics<1>[width=\textwidth]{Venezuela.png}}
%		\column{0.5\textwidth}
%			\parbox[c][0.9\textheight]{0.9\textwidth}
%			{\includegraphics<1>[width=\textwidth]{BrazilProtestjpg.jpg}}
%	\end{columns}
\end{frame}
%%%%%%%%%%%%%%%%%%%%%%%%%%%%%%%%%%%%%%%%%%%%%%%%%%%%%%
\begin{frame}
	\frametitle{Origins of the Project}
	Talk about LISA \\
	CMI \\
	VDGIF\\
	Relevance - of project	
\end{frame}
%%%%%%%%%%%%%%%%%%%%%%%%%%%%%%%%%%%%%%%%%%%%%%%%%%%%%%
\begin{frame}
	\frametitle{Policy Question}
	\begin{itemize}
		\item The clients were concerned with cruelty to the foxes.
		\item While we had no data concerning cruelty, we could tell when a fox died based on its transponder signal.
		\item Thus, we were in a position to investigate what influences fox mortality.
	\end{itemize}
\end{frame}
%%%%%%%%%%%%%%%%%%%%%%%%%%%%%%%%%%%%%%%%%%%%%%%%%%%%%%
%%%%%%%%%%%%%%%%%%%%%%%%%%%%%%%%%%%%%%%%%%%%%%%%%%%%%%
\section{Data Overview}
\begin{frame}
	\frametitle{Data Overview}
		\begin{itemize}
			\item Data was provided for trial days and included, fox tag IDs, the experience of the foxes, mortality data, and the number of dogs.
			\item For interim days, we knew when a fox died could infer their experience, but only had vague information about the number of dogs.
			\item We handled this limitation by taking a missing data approach to the analysis.
		\end{itemize}
	\end{frame}
%%%%%%%%%%%%%%%%%%%%%%%%%%%%%%%%%%%%%%%%%%%%%%%%%%%%%

\begin{frame}
	\frametitle{Missing Data}
	quick overview of missing data \\
	add math to show Data Augmentation to integrate out
	\end{frame}
%%%%%%%%%%%%%%%%%%%%%%%%%%%%%%%%%%%%%%%%%%%%%%%%%%%%%%
%%%%%%%%%%%%%%%%%%%%%%%%%%%%%%%%%%%%%%%%%%%%%%%%%%%%%%
\section{Modeling}
\subsection{Modeling Specification}
\begin{frame}
	\frametitle{Modeling Specification}
\begin{eqnarray}
y_{it} &\sim& Bernoulli(p_{it}) \\
probit(p_{it}) & = & \alpha + X_{it}\beta+ \theta_{it} \\
\theta_{it} &\sim& N(0, \phi^{-1})
\end{eqnarray}
where $X_{it} = [Num.dogs_t\; log.experience_{it}\; (num.dogs*log.experience)_{it}],$ $i = \{1,...27\} $(fox), and $t= \{1,...,T\}$ (trials). Let $R_{it}$ be the risk matrix, where
\[
    R_{it}=\left\{
                \begin{array}{ll}
                  1 \quad \text{ if fox i is alive and collared on day t-1}\\
                  0 \quad \text{otherwise}
                \end{array}
              \right.
  \]
 \end{frame}
%%%%%%%%%%%%%%%%%%%%%%%%%%%%%%%%%%%%%%%%%%%%%%%%%%%%%%
\begin{frame}
	\frametitle{Modeling Specification}
	 We use data augmentation where 
	 \[
    Y_{it}=\left\{
                \begin{array}{ll}
                  1 \quad  Z_{it} > 0\\
                  0 \quad Z_{it} \leq 0
                \end{array}
              \right.
  \]
and then
  \begin{eqnarray}
  Z_{it} \sim N( \alpha + X_{it}\beta +\theta_{it},1).
  \end{eqnarray}
\\
	discuss priors here and missing data piece
\end{frame}
%%%%%%%%%%%%%%%%%%%%%%%%%%%%%%%%%%%%%%%%%%%%%%%%%%%%%%
%%%%%%%%%%%%%%%%%%%%%%%%%%%%%%%%%%%%%%%%%%%%%%%%%%%%%%
\section{Results}
\subsection{Coefficient Estimates}
\begin{frame}
	\frametitle{Posterior Summaries}
\begin{table}[h]
%\caption{default}
\begin{center}
\begin{tabular}{|c|c|c|}
\hline
 & mean & CI \\
 \hline
  $\alpha$ & 5.4 & (3.0,11.6) \\
 $\beta_{dogs}$ & -.008 & (-.015, -.004) \\
 $\beta_{exp}$ & -.65 & (-2.04,-.10) \\
 $\beta_{dogs*exp}$ & .0014 & (.0006, .0030)\\
 \hline
\end{tabular}
\end{center}
\label{default}
\end{table}%

\end{frame}
%%%%%%%%%%%%%%%%%%%%%%%%%%%%%%%%%%%%%%%%%%%%%%%%%%%%%%
\subsection{Graphical Summaries}
\begin{frame}
	\frametitle{Response Surface}
	\begin{columns}
		\column{0.5\textwidth}
			\parbox[c][0.9\textheight]{0.9\textwidth}
				{\includegraphics<1>[width=\textwidth]{ResponseSurface.pdf}}
		\column{0.5\textwidth}
			\parbox[c][0.9\textheight]{0.9\textwidth}
			{\includegraphics<1>[width=\textwidth]{SurvivalProb.pdf}}
	\end{columns}

\end{frame}
%%%%%%%%%%%%%%%%%%%%%%%%%%%%%%%%%%%%%%%%%%%%%%%%%%%%%%
\begin{frame}
	\frametitle{Regime Study}
	All regimes assume a fox was placed in the pen on the first day of the study.
	\begin{itemize}
	\item Regime A: No constraints on number of dogs or allotted acclimation time.
	\item Regime B: Two weeks acclimation time with no dogs.
	\item Regime C: No more than 400 dogs allowed in pen at a time.
	\item Regime D: Two weeks acclimation time and 400 dog limit.
	\end{itemize}
%	At the finest level $p(\boldsymbol{Y}_{tL_j}|\boldsymbol{\mu}_{tL_j}) \overset{ind}{\sim} \text{Poisson}
\end{frame}


\begin{frame}
	\frametitle{Regime Survival Plots}
	\begin{columns}
		\column{0.5\textwidth}
			\parbox[c][0.9\textheight]{0.9\textwidth}
				{\includegraphics<1>[width=\textwidth]{RegA.pdf}}
						\column{0.5\textwidth}
			\parbox[c][0.9\textheight]{0.9\textwidth}
			{\includegraphics<1>[width=\textwidth]{RegB.pdf}}
	\end{columns}
\end{frame}

\begin{frame}
	\frametitle{Regime Survival Plots}
	\begin{columns}
		\column{0.5\textwidth}
			\parbox[c][0.9\textheight]{0.9\textwidth}
				{\includegraphics<1>[width=\textwidth]{RegC.pdf}}
						\column{0.5\textwidth}
			\parbox[c][0.9\textheight]{0.9\textwidth}
			{\includegraphics<1>[width=\textwidth]{RegD.pdf}}
	\end{columns}
\end{frame}

\begin{frame}
	\frametitle{Regime Analysis}
		\begin{itemize}
			\item Mortality is greatest with many dogs and for inexperienced foxes, but survival greatly improves either with a capped dog count or some time for the foxes to acclimate to the pen.
			\item Trials are run weekly, so two weeks acclimation is impractical. Thus, we can most practically smooth out fox mortality by capping the number of dogs in the pen.
		\end{itemize}
\end{frame}

\subsection{Final Thoughts}
\begin{frame}
	\frametitle{What's Ahead}
		While we have addressed fox mortality, it remains to make the connection between mortality and cruelty. 
\end{frame}

\end{document}

