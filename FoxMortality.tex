\documentclass[aoas,preprint]{imsart}

\RequirePackage[OT1]{fontenc}
\RequirePackage{amsthm,amsmath}
\RequirePackage{natbib}
\RequirePackage[colorlinks,citecolor=blue,urlcolor=blue]{hyperref}
\usepackage{graphicx}

% settings
%\pubyear{2005}
%\volume{0}
%\issue{0}
%\firstpage{1}
%\lastpage{8}
\arxiv{arXiv:0000.0000}

\startlocaldefs
\numberwithin{equation}{section}
\theoremstyle{plain}
\newtheorem{thm}{Theorem}[section]
\endlocaldefs

\begin{document}

\begin{frontmatter}
%\title{Assessing Fox Mortality during Fox Hunting Trials\thanksref{T1}}
\runtitle{Fox Mortality}
%\thankstext{T1}{Footnote to the title with the ``thankstext'' command.}

\begin{aug}
\author{\fnms{Andrew} \snm{Hoegh}\thanksref{m1}\ead[label=e1]{ahoegh@vt.edu}},
\author{\fnms{Ian} \snm{Crandell}\thanksref{m1}\ead[label=e2]{ian85@vt.edu}}


\thankstext{t1}{Some comment}
\runauthor{Hoegh \& Crandell}

\affiliation{Department of Statistics, Virginia Tech\thanksmark{m1}}

\address{Department of Statistics, Virginia Tech\\
Hutcheson Hall 401-D\\
Blacksburg, VA 24061\\
\printead{e1}\\
\phantom{E-mail:\ }\printead*{e2}}

\end{aug}

\begin{abstract}
Fox penning is a highly controversial practice whereby foxes are hunted with dogs in a large enclosed space. While there are commonalities with traditional horseback fox hunting, there are definite differences as well. The main difference is that with fox penning there is no chance for the fox to escape the enclosure, which seems to violate the idea of fair chase. Using data from a fox enclosure in Virginia, we investigate the factors that influence a fox's chances to survive in such pens. We then use our model to examine possible changes to fox penning policy and what effect those changes may have on fox survival. We conclude that by either allowing the foxes to acclimate to their environment or by limiting the number of hunting dogs in the pen, we can improve fox survival.
\end{abstract}

\begin{keyword}
\kwd{Survival Analysis}
\kwd{Missing Data}
\kwd{Bayesian Modeling}
\end{keyword}

\end{frontmatter}

\section{Introduction}

Fox penning is a highly charged practice, both in Virginia and nationally. A major issue revolves around the concept of \emph{fair chase} \citep{posewitz} and more broadly of ethics of the hunter. Fair chase is defined as: `the ethical, sportsmanlike, and lawful pursuit and taking of any free-ranging wild, native North American big game animal in a manner that does not give the hunter an improper advantage over such animals' \citep{boone}. One could argue maintaining animals in enclosures for hunting or training purposes, cannot satisfy the requirements of fair chase. Nevertheless, we focus on  assessing fox mortality on fox pens. By identifying and controlling factors contributing to fox survival, we can mimic the spirit of fair chase by enabling high survival rates of foxes.

As with any controversial practice, there are strong opinions on both sides of the issue. With this article we stick to objective, measurable characteristics like survival rate rather than delving into subjective assessments of ethics. With that in mind, we use data from a fox pen in Virginia to assess survival of foxes in enclosures. Specifically we focus on the effect of dogs in the pens and the time a fox has to acclimate to the new surroundings. Finally, a note of caution is necessary for these results. The data is not a product of a well designed experimental trial across several fox enclosures, but rather is collected at a single site. As such, caution should be exhibit when applying results from this pen to other enclosures, both in Virginia and nationally. Nevertheless, our assessment contains data driven results for evaluating policy implications.

The remainder of this article follows as: Section 2 outlines the data in the study, Section 3 contains details on modeling, Section 4 describes the model results, Section 5 presents a set of hypothetical policy regimes, and Section 6 concludes with a discussion.
\section{Data} Upon being placed in the fox enclosure foxes were equipped with radio transmitters. This study contains information from 27 foxes over the course of 17 months from October 2002 to February 2004. The study period contains two distinct types of days: trial days and non-trial days. Trial days consist of competitions where large groups of dogs are brought to the enclosures by owners. On trial days there is a record of the number of dogs in the pen. The number of dogs ranges from 50 to 752 during the trials. On non-trial days the number of dogs is not available. Unfortunately, this does not imply that no dogs are present. Rather, the number of dogs is typically non-zero and tends to be less than fifty.

The foxes are not all placed into the pen at the start of the study period, instead they are placed into the enclosure throughout the study period. Of the 27 foxes, 6 survive to the end of the study. Figure \ref{fig:SurvTime} contains a box plot along with points for each fox of the survival time for the foxes that perished during the study period. 
\begin{figure}[h!]
\begin{center}
\caption{Survival Time of Foxes, dots denote time in enclosure at death and 'x' denotes time in enclosure at end of study}
\includegraphics[width=.7\textwidth]{Survival_Time.pdf}
\label{fig:SurvTime}
\end{center}
\end{figure}
Additionally this figure also contains how long each fox has been in the enclosure for the foxes that were alive at the end of the trial period. We consider the number of dogs in the enclosure and the time a fox has had to acclimated to the enclosure as the factors for fox survival. Obviously more dogs hunting the foxes should lower the survival rate. Acclimation time allows a fox to establish a territory and become acquainted with the area and perhaps find dens or other features that increase survival probabilities.

\subsection{Missing Data}
As the number of dogs in the enclosure is unknown and not strictly zero, our models need to account for this fact. Simply using zeros (or any other constant) would fail to capture the uncertainty present in the data and lead to flawed inferences. Generally speaking this is known as missing data \citep{little}. A common classical technique for dealing with missing data is multiple imputation \citep{rubin}. In essence, imputation is conducted several times to consider the uncertainty inherent in predicting unknown values.

From a Bayesian perspective, inferences are made from the posterior distribution, $P(\Theta|X,Y).$ With missing covariate data consider $
	\mathcal{X}=\left[
	\begin{array}{ll}
	X \\
	X^{*} 
	\end{array}
	\right]$ \\ 
	and $X^{*}$ denotes missing covariates. Then in given a prior distribution on the missing covariates, uncertainty in the missing data is accounted for in a typical Bayesian framework by integrating over the distribution \citep{boone2009}. Specifically, the posterior distribution can now be computed as $$P(\Theta|X,Y), = \int P(\Theta|X,X^{*},Y) p(X^{*}) dX^{*}.$$

\section{Model} While traditional survival models such as the Cox proportional hazards model \citep{cox} or a discrete survival model were considered, this scenario does not fit neatly into that framework. In particular, we are concerned with policy implications for survival of foxes in enclosures. For policy considerations, it is more practical to impose restrictions on daily (or annual behaviors) than to consider over the lifetime of a fox. With this said, we consider a model for the survival of fox on a given day based on the number of dogs in the enclosure and the acclimation time of the fox.

Specifically a binary regression framework is invoked using a generalized linear model with a probit link function.
\begin{eqnarray}
y_{it} &\sim& Bernoulli(p_{it}) \\
probit(p_{it}) & = & \alpha + \mathcal{X}_{it}\boldsymbol{\beta} + \theta_{i} \\
\theta_{i} &\sim& N(0, \phi^{-1})
\end{eqnarray}
where $i = \{1,...27\} $(fox), and $t= \{1,...,T\}$ (time). The variable $y_{it}$ is a binary variable corresponding to survival($=1$) or death of fox $i$ on day $t$. The matrix $\mathcal{X}_{it} = [Dogs_t\; log.exp_{it}\; (Dogs*log.exp)_{it}].$ The $\theta_i$ terms are random effects for each fox. Let $R_{it}$ be the risk matrix, where
\[
    R_{it}=\left\{
                \begin{array}{ll}
                  1 \quad \text{ if fox i is alive and collared on day t-1}\\
                  0 \quad \text{otherwise}
                \end{array}
              \right.
  \]
  As in \cite{albert}, we use data augmentation where 
  \begin{eqnarray}
  Z_{it} \sim N( \alpha + X_{it}\boldsymbol{\beta} +\theta_{i})
  \end{eqnarray}
  and
\[
    Y_{it}=\left\{
                \begin{array}{ll}
                  1 \quad  Z_{it} > 0\\
                  0 \quad Z_{it} \leq 0
                \end{array}
              \right.
  \]
  
  For prior specification, we use conjugate priors: $p(\alpha) \sim N(\alpha;0,1), \boldsymbol{\beta} \sim N(\boldsymbol{\beta};0,\Sigma),$ where $\Sigma$ is a diagonal matrix scaled such that $\sigma_{ii}$ is twice the range of the data. An informative prior $\Gamma(5,5)$ is placed on $\phi$. This insures that the fox random effects are well behaved and do not drift off to $\infty$ as might be expected for foxes that survive all of their trials. This analysis also requires priors on the missing data, the number of dogs on non-trial days. With this analysis we believe that the number of dogs is generally less than 50, so a truncated normal prior is used. Specifically $p(X^{*}) \sim N(X^{*}; 0 ,10^2,0,50)$.
  
  ADD LIKELIHOOD OR POSTERIOR MATH HERE

  
\section{Results}
Using conjugate priors enables Gibbs sampling for each of the parameters. The sampler was run for 500,000 iterations to insure convergence for all of the parameters. Estimates of the coefficients and 95\% credible intervals are give in Table \ref{tab:PostEst}. 
\begin{table}[h!]
	\begin{center}
		\begin{tabular}{|c|c|c|}
			\hline
			& mean & CI \\
			\hline
			$\alpha$ & 5.4 & (3.0,11.6) \\
			$\beta_{dogs}$ & -.008 & (-.015, -.004) \\
			$\beta_{exp}$ & -.65 & (-2.04,-.10) \\
			$\beta_{dogs*exp}$ & .0014 & (.0006, .0030)\\
			\hline
		\end{tabular}
		\label{tab:PostEst}
	\end{center}
	\caption{Table of model coefficients.}
\end{table}
It can be seen that no credible interval contains 0, indicating that all predictors are useful. Interpretation of individual coefficients is complicated by the presence of the interaction term, but we can understand the model output by examining the heat map in Figure \ref{fig:SurvProb2}. 
\begin{figure}[h!]
\begin{center}
\caption{Survival Probability of Foxes as a function of experience and dogs in enclosure.}
\includegraphics[width=.7\textwidth]{survivalprob.pdf}
\label{fig:SurvProb2}
\end{center}
\end{figure}
We see that fox survival is lowest when there are many dogs and inexperienced foxes. This survival rises drastically as either the number of dogs drops or the experience of the foxes increases.

\section{Policy Analysis}

\paragraph{}One of the motivating goals of this paper was to find ways to make fox penning less cruel. To learn about this, we considered how fox survival might change if the pens were to make policy changes. We considered four different situations:

	\begin{itemize}
		\item Regime A: No constraints on number of dogs or allotted acclimation time. This is currently in place.
		\item Regime B: Two weeks acclimation time with no dogs.
		\item Regime C: No more than 400 dogs allowed in pen at a time.
		\item Regime D: Two weeks acclimation time and 400 dog limit.
	\end{itemize}
These four regimes produced the following survival curves.

\begin{figure}[h]
	\begin{center}$
		\begin{array}{cc}
		\includegraphics[width=.5\textwidth]{RegA.pdf} & \includegraphics[width=.5\textwidth]{RegB.pdf}\\
		\includegraphics[width=.5\textwidth]{RegC.pdf} & \includegraphics[width=.5\textwidth]{RegD.pdf}
		\end{array}$
	\end{center}
	\caption{Survival functions for the four regimes}
	\label{fgr:1}
\end{figure}
We see a precipitous drop in fox survival for regime A but not for any of the other three regimes. In fact, the other three plots look very similar.

\subsection{study limitations}
We do not do a comparison of survival rates in the wild generally considered to be around five years \citep{hunter}, as the ages of the foxes place in the enclosure are unknown.

This is not a designed experiment, where inferences naturally apply to other enclosures as well...
\section{Discussion}
While our analysis has been about fox survival, what we really wanted to learn about was cruelty. Indeed, it is not the intent of the pen that the fox should survive, but rather than the fox be given a fair chance at evasion in accordance with the concept of the fair chase. According to our analysis, fewer than half of the foxes survive beyond their first day under the current rules system. This can hardly be said to be a fair chase. Fortunately, it only takes a small change in policy to greatly improve fox survival. By either allowing the foxes acclimation time or by limiting the number of dogs in the pen we can eliminate the drastic fox mortality we see in the current regime and more closely align the practice of fox penning with the idea of the fair chase.









\bibliographystyle{asa}
\bibliography{refsFox}

%\begin{thebibliography}{9}
%
%\bibitem{r1}
%\textsc{Billingsley, P.} (1999). \textit{Convergence of
%Probability Measures}, 2nd ed.
%Wiley, New York.
%\MR{1700749}
%
%
%\end{thebibliography}

\end{document}
