\documentclass[aoas,preprint]{imsart}

\RequirePackage[OT1]{fontenc}
\RequirePackage{amsthm,amsmath}
\RequirePackage[numbers]{natbib}
\RequirePackage[colorlinks,citecolor=blue,urlcolor=blue]{hyperref}

% settings
%\pubyear{2005}
%\volume{0}
%\issue{0}
%\firstpage{1}
%\lastpage{8}
\arxiv{arXiv:0000.0000}

\startlocaldefs
\numberwithin{equation}{section}
\theoremstyle{plain}
\newtheorem{thm}{Theorem}[section]
\endlocaldefs

\begin{document}

\begin{frontmatter}
\title{Assessing Fox Mortality during Fox Hunting Trials\thanksref{T1}}
\runtitle{Fox Mortality}
%\thankstext{T1}{Footnote to the title with the ``thankstext'' command.}

\begin{aug}
\author{\fnms{First} \snm{Author}\thanksref{m1}\ead[label=e1]{first@somewhere.com}},
\author{\fnms{Second} \snm{Author}\thanksref{m1}\ead[label=e2]{second@somewhere.com}}


\thankstext{t1}{Some comment}
\runauthor{F. Author et al.}

\affiliation{Virginia Polytechnic Institute and State University\thanksmark{m1}}

\address{Address of the First and Second authors\\
Usually a few lines long\\
\printead{e1}\\
\phantom{E-mail:\ }\printead*{e2}}

\end{aug}

\begin{abstract}
The abstract should summarize the contents of the paper.
It should be clear, descriptive, self-explanatory and not longer
than 200 words. It should also be suitable for publication in
abstracting services. Please avoid using math formulas as much as possible.

\end{abstract}

\begin{keyword}[class=MSC]
\kwd[Primary ]{60K35}
\kwd{60K35}
\kwd[; secondary ]{60K35}
\end{keyword}

\begin{keyword}
\kwd{sample}
\kwd{\LaTeXe}
\end{keyword}

\end{frontmatter}

\section{Introduction}
test...
\section{Data}

\section{Model}

\begin{eqnarray}
y_{it} &\sim& Bernoulli(p_{it}) \\
probit(p_{it}) & = & \alpha + X_{t}\beta_{dogs} + \theta_{it} \\
\theta_{it} &\sim& N(\alpha_{i} + G_{it} \beta_{exp}, \phi^{-1})
\end{eqnarray}
where $i = \{1,...27\} $(fox), and $t= \{1,...,T\}$ (trials). The number of dogs for trial t is denoted $X_t$ and $G_{it}$ is the experience of fox $i$ on day $t$. Let $R_{it}$ be the risk matrix, where
\[
    R_{it}=\left\{
                \begin{array}{ll}
                  1 \quad \text{ if fox i is alive and collared on day t-1}\\
                  0 \quad \text{otherwise}
                \end{array}
              \right.
  \]
  We use data augmentation where 
  \begin{eqnarray}
  Z_{it} \sim N( \alpha + X_t\beta_{dogs} +\theta_{it}
  \end{eqnarray}
  and
\[
    Y_{it}=\left\{
                \begin{array}{ll}
                  1 \quad  Z_{it} > 0\\
                  0 \quad Z_{it} \leq 0
                \end{array}
              \right.
  \]
  
  \subsection{Outstanding Model Based Questions}
  \begin{itemize}
  \item Potential Interaction between experience and dogs
  \item how to handle non-trial deaths
  \begin{enumerate}
  \item exclude
  \item average dogs across week
  \item ???
  \end{enumerate}
  \item assess linearity of `XB' inside link function
  \item perhaps look at a comparison of the hazard (survival curve) for trial and non-trial days.
  \end{itemize}
  
\section{Results}
The goal is to learn $p(\beta_{dogs}|\tilde{y})$ and $p(\beta_{dogs}|\tilde{y})$ in order to assess the discrete hazard.
\section{Discussion}

\newpage



\begin{thebibliography}{9}

\bibitem{r1}
\textsc{Billingsley, P.} (1999). \textit{Convergence of
Probability Measures}, 2nd ed.
Wiley, New York.
\MR{1700749}


\end{thebibliography}

\end{document}
